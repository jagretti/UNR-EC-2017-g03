\subsection{Operation Model for maxHandlingDelayPassed}

\label{OM-maxHandlingDelayPassed}


The \msrcode{maxHandlingDelayPassed} operation has the following properties:

	\begin{operationmodel}
	\addheading{Operation}
	\adddoublerow{maxHandlingDelayPassed}{used to determine if the crisis stood too longly in a pending status since its creation.}


	\addrowheading{Return type}
	\addsinglerow{ptBoolean}

		

	\addrowheading{Post-Condition (functional)}
	\addnumberedsinglerow{PostF}{ true iff the crisis is in pending status and if the duration between the current ctState clock information and the crisis instant is greater than the maximum reminder period duration.}

	\end{operationmodel}



	% ------------------------------------------
	% MCL Listing
	% ------------------------------------------
	\vspace{1cm}
	The listing~\ref{OM-ctCrisis-maxHandlingDelayPassed-MCL-LST} provides the \msrmessir (MCL-oriented) specification of the operation.
	
	\scriptsize
	\vspace{0.5cm}
	\begin{lstlisting}[style=MessirStyle,firstnumber=auto,captionpos=b,caption={\msrmessir (MCL-oriented) specification of the operation \emph{maxHandlingDelayPassed}.},label=OM-ctCrisis-maxHandlingDelayPassed-MCL-LST]

	
	
	/* Post Functional:*/ 
	postF{let TheSystem:ctState in
	let CurrentClockSecondsQty:dtInteger in
	let CrisisInstantSecondsQty:dtInteger in
	let MaxCrisisReminderPeriod:dtSecond in
	if 
	( /* Post F01 */
	  self.rnSystem = TheSystem
	  and self.status = pending
	  and TheSystem.clock.toSecondsQty() = CurrentClockSecondsQty
	  and Self.instant.toSecondsQty() = CrisisInstantSecondsQty
	  and TheSystem.maxCrisisReminderPeriod = MaxCrisisReminderPeriod
	  and CurrentClockSecondsQty.sub(CrisisInstantSecondsQty)
	                            .gt(MaxCrisisReminderPeriod)
	)
	then (result = true)
	else (result = false)
	endif}
	
	
	\end{lstlisting}
	\normalsize 
	
	
	
	





