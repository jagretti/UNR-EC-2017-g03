\subsection{Operation Model for oeValidateAlert}

\label{OM-oeValidateAlert}


The \msrcode{oeValidateAlert} operation has the following properties:

	\begin{operationmodel}
	\addheading{Operation}
	\adddoublerow{oeValidateAlert}{sent to indicate that a specific alert is not a fake.}

	\addrowheading{Parameters}
	\addnumbereddoublerow{}{AdtAlertID: dtAlertID}{the identification information used to determine the alert instance} 

	\addrowheading{Return type}
	\addsinglerow{ptBoolean}

	\addrowheading{Pre-Condition (protocol)}
	\addnumberedsinglerow{PreP}{the system is started}
	\addnumberedsinglerow{PreP}{the actor logged previously and did not log out ! (i.e. the associated ctCoordinator instance is considered logged)}
		
	\addrowheading{Pre-Condition (functional)}
	\addnumberedsinglerow{PreF}{ it is supposed that there exist one ctAlert instance with the same \msrcode{id} attribute value as the one provided by the coordinator actor who wants to validate.}

	\addrowheading{Post-Condition (functional)}
	\addnumberedsinglerow{PostF}{the ctAlert class instance having the provided id is considered as valid in the post state and the coordinator actor is informed about the satisfaction of its request.}
	\addnumberedsinglerow{PostF}{If the hash of the actual alert it is not equal to the hash of the backup, then the actual alert its replaced 
	by the backup alert, because the integrity has been corrupted}

	\addrowheading{Post-Condition (protocol)}
	\addnumberedsinglerow{PostP}{ none}
	\end{operationmodel}



	% ------------------------------------------
	% MCL Listing
	% ------------------------------------------
	\vspace{1cm}
	The listing~\ref{OM-actCoordinator-oeValidateAlert-MCL-LST} provides the \msrmessir (MCL-oriented) specification of the operation.
	
	\scriptsize
	\vspace{0.5cm}
	\begin{lstlisting}[style=MessirStyle,firstnumber=auto,captionpos=b,caption={\msrmessir (MCL-oriented) specification of the operation \emph{oeValidateAlert}.},label=OM-actCoordinator-oeValidateAlert-MCL-LST]

	/* Pre Protocol:*/ 
	preP{let theSystem: ctState in
	  		let theActor:actAuthenticated in
	  		self.rnActor.rnSystem = TheSystem
	  		and self.rnActor = TheActor
	  
			/* PreP01 */
	  		and TheSystem.vpStarted = true
			/* PreP02 */
	  		and TheActor.rnctAuthenticated.vpIsLogged = true}
	
	/* Pre Functional:*/
	preF{let theSystem: ctState = self.rnActor.rnSystem in
	  		let theAlerts: Set(ctAlert) = theSystem.rnctAlert in
	  		theAlerts -> exists(a: ctAlert | a.id = AdtAlertID)}
	
	/* Post Functional:*/ 
	postF{let theSystem: ctState in
	  		let theAlerts: Set(ctAlert) in
	  		let theAlert: ctAlert in
	  		let theAlertsBackup: Set(ctAlertBackup) in
	  		let theAlertBackup: ctAlertBackup in
	  		
	  		theSystem = self.rnActor.rnSystem and
	  		theAlerts = theSystem.rnctAlert and
	  		theAlertsBackup = theSystem.rnctAlertBackup and
	  		theAlerts->any(a: ctAlert | a.id = AdtAlertID) = theAlert and
	  		theAlertsBackup->any(a: ctAlertBackup | a.id = AdtAlertID) = theAlertBackup and
	  		
	  		// Checking integrity
	  		if (theAlert.hash.isEqual(theAlertBackup.hash))
	  		then (theAlert@post.status = "valid")
	  		else (theAlert@post = theAlertBackup
	  			and theAlert@post.status = "valid"
	  			and theAlerts@post = (theAlerts.excluding(theAlert)).including(theAlert@post)
	  		)
	  		endif
	  		self.rnActor.rnInterfaceIN^ieValidateAlert(theAlert.id)}
	
	/* Post Protocol:*/ 
	postP{ true}
	
	\end{lstlisting}
	\normalsize 
	
	
	
	





